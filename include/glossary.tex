\newacronym{vm}{VM}{Virtual~Machine}
\newacronym{ha}{HA}{High~Availability}
\newacronym{lvm}{LVM}{Logical Volume Manager}
\newacronym{kvm}{KVM}{Kernel-based Virtual Machine}
\newacronym{api}{API}{Application Programming Interface}
\newacronym{pv}{PV}{Physical Volume}
\newacronym{vg}{VG}{Volume Group}
\newacronym{lv}{LV}{Logical Volume}
\newacronym{ssh}{SSH}{Secure Shell}
\newacronym{vnc}{VNC}{Virtual Network Computing}
\newacronym{rpm}{RPM}{RPM Package Manager, eredetileg Red Hat Package Manager}
\newacronym{os}{OS}{Operating System}
\newacronym{sle}{SLE}{SUSE Linux Enterprise}
\newacronym{cpu}{CPU}{Central Processing Unit}
\newacronym{ssd}{SSD}{Solid State Drive}
\newacronym{spice}{SPICE}{Simple Protocol for Independent Computing Environments}
\newacronym{dns}{DNS}{Domain Name System}
\newacronym{smtp}{SMTP}{Simple Mail Transfer Protocol}
\newacronym{raid}{RAID}{Redundant Array of Independent Disks}

\newglossaryentry{hypervisor}
{
	name=hypervisor,
	description={Az a szoftver, amely koordinálja a virtuális gépek és az azokat futtató fizikai gép hardvere közötti interakciót}
}

\newglossaryentry{hotswap}
{
	name={hot swap},
	description={Számítógép-komponensek eltávolítása vagy hozzáadása egy futó rendszerhez annak leállítása vagy újraindítása nélkül}
}

\newglossaryentry{libvirt}
{
	name=libvirt,
	description={Nyílt forráskódú virtualizációs \acrshort{api}, mely lehetőséget biztosít több \gls{hypervisor} (pl.~\acrshort{kvm}, XEN, VMware ESXi) egységes interfészen keresztüli kezelésére}
}

\newglossaryentry{overcommit}
{
	name=overcommit,
	description={Hatékonyabb erőforráskihasználást lehetővé tevő technológia egyes \gls{hypervisor}okban. Használatával a virtuális gépek számára több virtuális erőforrást oszthatunk ki, mint amennyi ténylegesen rendelkezésre áll a gazdagépen. Alapvetően előnyös lehet a használata, mivel a \acrshort{vm}-ek általában nem használják ki a nekik biztosított erőforrások 100\%-át, de a gazdagép túlterhelése esetén jelentős teljesítményvisszaeséssel járhat, ezért fontos, hogy használata esetén még körültekintőbben monitorozzuk rendszereinket~\cite{RedHatKvmOvercommit}}
}

\newglossaryentry{highstate}
{
	name=highstate,
	description={A Salt keretrendszerben egy olyan speciális state, mely magában foglalja egy adott klienshez vagy egy klienscsoporthoz rendelt összes state beállítását. A highstate végrehajtása során a rendszer megpróbálja a state fájlokban meghatározottakkal összhangba hozni a kiválasztott klienseket}
}
