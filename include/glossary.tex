\newacronym{vm}{VM}{virtual~machine}
\newacronym{ha}{HA}{high~availability}
\newacronym{lvm}{LVM}{Logical Volume Manager}
\newacronym{kvm}{KVM}{Kernel-based Virtual Machine}
\newacronym{api}{API}{application programming interface}
\newacronym{pv}{PV}{physical volume}
\newacronym{vg}{VG}{volume group}
\newacronym{lv}{LV}{logical volume}
\newacronym{ssh}{SSH}{secure shell}
\newacronym{vnc}{VNC}{virtual network computing}
\newacronym{rpm}{RPM}{RPM Package Manager, eredetileg Red Hat Package Manager}
\newacronym{os}{OS}{operating system}
\newacronym{sle}{SLE}{SUSE Linux Enterprise}

\newglossaryentry{hypervisor}
{
	name=hypervisor,
	description={Az a szoftver, amely koordinálja a virtuális gépek és az azokat futtató fizikai gép hardvere közötti interakciót}
}

\newglossaryentry{hotswap}
{
	name={hot swap},
	description={Számítógép-komponensek eltávolítása vagy hozzáadása egy futó rendszerhez (annak leállítása vagy újraindítása nélkül)}
}

\newglossaryentry{libvirt}
{
	name=libvirt,
	description={Nyílt forráskódú virtualizációs \acrshort{api}, mely lehetőséget biztosít több \gls{hypervisor} (pl. \acrshort{kvm}, XEN, VMware ESXi) egységes interfészen keresztüli kezelésére}
}