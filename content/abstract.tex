\pagenumbering{roman}
\setcounter{page}{1}

\selecthungarian

%----------------------------------------------------------------------------
% Abstract in Hungarian
%----------------------------------------------------------------------------
\chapter*{Kivonat}\addcontentsline{toc}{chapter}{Kivonat}

Az informatika életünk szerves részévé vált: az interneten keresztül vásárolhatunk, intézhetjük banki ügyeinket és kommunikálhatunk egymással. Az egyre szélesebb körben elérhető szolgáltatások folyamatos üzemeltetése alapos tervezést, nagy szakértelmet és felkészültséget kíván meg a rendszermérnököktől.

A megnövekedett forgalom hatékony kiszolgálásához virtualizációs technológiákat alkalmaznak. Ezek lehetővé teszik, hogy számos különféle szolgáltatást tudjanak egy-egy szervergépen futtatni kedvező feltételek mellett: a virtualizáció csökkenti a szolgáltatások futtatásához szükséges helyigényt, és jobb erőforrás-kihasználtságot eredményez.

Az így létrejövő nagyméretű -- akár több ezer számítógépből álló -- infrastruktúrák karbantartása, üzemeltetése kézi megoldásokkal nehezen kivitelezhető. Emiatt jellemzően valamilyen infrastruktúra-menedzsment megoldást használnak a szervezetek, melyekkel központosítottan, egyetlen felületről van lehetőségük az üzemeltetett szerverek kezelésére, azok konfigurációjára.

A megfelelő minőségű üzemeltetéshez elengedhetetlen a számítógépek és az azokon futó szolgáltatások folyamatos felügyelete, monitorozása. Ennek segítségével figyelemmel követhetjük rendszereink állapotát, megfelelő határértékek beállításával pedig időben értesülhetünk az esetleges kialakulóban lévő problémákról, ezzel lehetőséget adva a hibák idejekorán történő javítására és a szolgáltatáskiesések megelőzésére.

Dolgozatomban a nagyvállalatokra jellemző informatikai környezeteket vizsgálom meg egy kis méretű tesztrendszeren keresztül. A tesztkörnyezetben kitérek a fent ismertetett technológiák alkalmazására, az ezek használata során követendő jó gyakorlatokra, és az általam tapasztalt kihívásokra, nehézségekre.


\vfill
\selectenglish


%----------------------------------------------------------------------------
% Abstract in English
%----------------------------------------------------------------------------
\chapter*{Abstract}\addcontentsline{toc}{chapter}{Abstract}

Information Technology (IT) has become an integral part of our lives, enabling us to shop, bank, and communicate online. The non-stop operation of an ever-wider range of services requires careful planning, expertise, and a high level of preparedness on the part of systems engineers.

Virtualisation technologies are used to efficiently serve the increased traffic. These allow a wide variety of services to run on a single server under favourable conditions: virtualisation reduces the space required to run services and results in better resource utilisation.

The resulting large-scale infrastructures -- up to thousands of computers -- are difficult to maintain and operate manually. For this reason, organisations typically use some kind of infrastructure management system that allows them to manage and configure the servers they operate from a centralised, single interface.

Continuous supervision and monitoring of computers and the services running on them is essential for ensuring the quality of operations. This allows us to monitor the status of our systems and, by setting appropriate thresholds, to be notified in time of any problems that may arise, thus providing the opportunity to correct errors in a timely manner and prevent service outages.

In my thesis, I examine IT environments typical of large enterprises through a small-scale test system. In the test environment, I will discuss the use of the technologies described above, the best practices to follow when using them, and the challenges and difficulties I have encountered throughout my work.


\vfill
\selectthesislanguage

\newcounter{romanPage}
\setcounter{romanPage}{\value{page}}
\stepcounter{romanPage}