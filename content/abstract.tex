\pagenumbering{roman}
\setcounter{page}{1}

\selecthungarian

%----------------------------------------------------------------------------
% Abstract in Hungarian
%----------------------------------------------------------------------------
\chapter*{Kivonat}\addcontentsline{toc}{chapter}{Kivonat}

Az informatika életünk szerves részévé vált: az interneten keresztül vásárolhatunk, intézhetjük banki ügyeinket és kommunikálhatunk egymással. Az egyre szélesebb körben elérhető szolgáltatások folyamatos üzemeltetése alapos tervezést, nagy szakértelmet és felkészültséget kíván meg a rendszermérnököktől.

A megnövekedett forgalom hatékony kiszolgálásához virtualizációs technológiákat alkalmaznak. Ezek lehetővé teszik, hogy számos különféle szolgáltatást tudjanak egy-egy szervergépen futtatni kedvező feltételek mellett: a virtualizáció csökkenti a szolgáltatások futtatásához szükséges helyigényt, és jobb erőforrás-kihasználtságot eredményez.

Az így létrejövő nagyméretű -- akár több ezer számítógépből álló -- infrastruktúrák karbantartása, üzemeltetése kézi megoldásokkal nehezen kivitelezhető. Emiatt jellemzően valamilyen infrastruktúra-menedzsment megoldást használnak a szervezetek, melyekkel központosítottan, egyetlen felületről van lehetőségük az üzemeltetett szerverek kezelésére, azok konfigurációjára.

A megfelelő minőségű üzemeltetéshez elengedhetetlen a számítógépek és az azokon futó szolgáltatások folyamatos felügyelete, monitorozása. Ennek segítségével figyelemmel követhetjük rendszereink állapotát, megfelelő határértékek beállításával pedig időben értesülhetünk az esetleges kialakulóban lévő problémákról, ezzel lehetőséget adva a hibák javítására és a szolgáltatáskiesések megelőzésére.

Dolgozatomban a nagyvállalatokra jellemző informatikai környezeteket vizsgálom meg egy kisebb léptékű tesztrendszeren keresztül. A tesztkörnyezetben kitérek a fent ismertetett technológiák alkalmazására, az ezek használata során követendő jó gyakorlatokra, és az általam tapasztalt kihívásokra, nehézségekre.


\vfill
\selectenglish


%----------------------------------------------------------------------------
% Abstract in English
%----------------------------------------------------------------------------
\chapter*{Abstract}\addcontentsline{toc}{chapter}{Abstract}

This document is a \LaTeX-based skeleton for BSc/MSc~theses of students at the Electrical Engineering and Informatics Faculty, Budapest University of Technology and Economics. The usage of this skeleton is optional. It has been tested with the \emph{TeXLive} \TeX~implementation, and it requires the PDF-\LaTeX~compiler.


\vfill
\selectthesislanguage

\newcounter{romanPage}
\setcounter{romanPage}{\value{page}}
\stepcounter{romanPage}