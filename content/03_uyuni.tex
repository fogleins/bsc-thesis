% !TeX spellcheck = hu_HU
% !TeX encoding = UTF-8

\chapter{Infrastruktúramenedzsment: Uyuni}
Nagyméretű informatikai infrastruktúra kezelése esetén elengedhetetlen valamiféle infrastruktúramenedzsment eszköz használata. Ez nem csak könnyebbé teszi az üzemeltetést, de számos kiegészítő funkcióval is rendelkezik, lényegében egy helyen láthatunk minden releváns adatot, és egyazon felületről van lehetőségünk frissítések telepítésére és biztonsági sérülékenységek leírásainak böngészésre, mint ahol azt is tároljuk, hogy egy adott virtuális gép melyik gazdagépen fut, és az hol található.

Mivel ez a megoldás hasznosnak és érdekesnek tűnt számomra, és a dolgozat profiljába is jól illeszkedik, úgy döntöttem, hogy a tesztkörnyezet kezelésére is fogok ilyen megoldást alkalmazni. A választásom az Uyuni-ra esett, mely gyakorlatilag mindent tud, amire szükségem volt, és ingyenesen elérhető. A döntésben az is segített, hogy a projektet a SUSE támogatja, valamint a fejlesztésében is részt vesz, és az openSUSE-alapú disztribúciókra is kiemelt figyelmet fordítanak, így nem kellett kompatibilitási problémákkal foglalkoznom. Az Uyuni a Salt konfigurációmenedzsment és automatizációs keretrendszerre épül, mely szintén széleskörűen használt és támogatott.