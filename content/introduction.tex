% !TeX encoding = UTF-8
% !TeX spellcheck = hu_HU
%----------------------------------------------------------------------------
\chapter{\bevezetes}
%----------------------------------------------------------------------------

%A bevezető tartalmazza a diplomaterv-kiírás elemzését, történelmi előzményeit, a feladat indokoltságát (a motiváció leírását), az eddigi megoldásokat, és ennek tükrében a hallgató megoldásának összefoglalását.
%
%A bevezető szokás szerint a diplomaterv felépítésével záródik, azaz annak rövid leírásával, hogy melyik fejezet mivel foglalkozik.

Napjainkban az informatika és az internet életünk szerves részévé vált. A számos szolgáltatás folyamatos rendelkezésre állásának biztosítása és a megnövekedett forgalom kiszolgálása jó néhány új technológia kifejlesztését követelte meg.

Dolgozatomban elsősorban egy általános képet szeretnék adni arról, hogy milyen üzemeltetési kihívásokkal kell szembenéznünk, ha egy ilyen szolgáltatás működtetésébe vágjuk a fejszénket. Értekezésemben nem fogok kitérni bizonyos infrastrukturális hátterekre -- mint például a kiszolgálók folyamatos energiaellátásának biztosítása --, ezeket adottnak fogom tekinteni, hiszen ezt egy adatközpontban bérelt hely esetén sem magunknak kell biztosítanunk. A következőkben sokkal inkább az informatikai lehetőségek tárgyalására fogom helyezni a hangsúlyt: hogyan tudunk hatékonyan üzemeltetni több kiszolgálót, milyen módon lehet biztosítani a szolgáltatásaink lehető legnagyobb rendelkezésre állását, és hogyan védhetjük meg adatainkat egy esetlegesen félresikerült rendszerfrissítést követően. Mindezt egy kisebb volumenű tesztrendszerben is be fogom mutatni.

\section{Nagyvállalati környezetek ismertetése}
A legtöbb hétköznapi felhasználó számára ismeretlen vagy meglepő lehet, hogy maga az internet és az ezen keresztül elérhető szolgáltatások -- gondoljunk például az Ügyfélkapura vagy az internetbank-szolgáltatásokra -- nagyon komplex rendszerek nem csak szoftveres, hanem informatikai infrastruktúra szempontjából is. A legtöbb ilyen szolgáltatás egy adatközpontban lévő szerveren fut, ami a beérkező kérésekre ad válaszokat. Ezt a folyamatot úgy is felfoghatjuk, hogy az ilyen szolgáltatások felhasználói lényegében az adott szolgáltató (a fenti példánál maradva a Magyar Állam és az adott bankok) számítógépeivel kommunikálnak.

Ezek a szervergépek több lényeges különbséggel is bírnak a személyi számítógépekkel szemben. Egyik legfontosabb tulajdonságuk, hogy hibatűrőek bizonyos hardverhibákat illetően: szinte minden főbb komponensből legalább kettő áll rendelkezésre, így ha az egyik meg is hibásodik, akkor a hiba elhárításáig a beépített redundancia miatt a gép képes tovább funkcionálni, általában a felhasználók felé észrevétlenül, míg a gép üzemeltetői figyelmeztetést kapnak a hiba típusáról és a kapcsolódó tennivalókról.
% TODO: kép redundáns dolgokról (pl. hálózati kártya, PSU), és hibajelző elemekről (pl. CPU fault, RAM-hiba, diszk hiba)

\section{Virtualizáció}
A fent említett megnövekedett forgalom kiszolgálását hatékonyan lehet kezelni úgy, hogy olyan fizikai számítógépet helyezünk üzembe, mely  több, egymástól független operációs rendszer futtatására is alkalmas. Ilyenkor ezeket a fizikai gépen futó rendszereket virtuális gépeknek (virtual machine, VM) % TODO: rövidítésjegyzékbe
nevezzük. Egy virtuális gép elkülönített erőforrásokat kap a fizikai géptől, hozzáférhet például bizonyos mennyiségű processzormaghoz, memóriához, illetve külön háttértár-partíciói is lehetnek. A virtualizált hardverek és operációs rendszerek a legtöbb esetben a külvilág felé nem különböztethetőek meg a fizikai számítógépektől, és ezzel a megoldással jelentősen csökkenthető a rendszerek és a hozzájuk szükséges informatikai infrastruktúra üzemeltetésének költsége.
% TODO: virt-manager screenshot, esetleg virsh xml screenshot


\subsection{Teljes virtualizáció és paravirtualizáció összehasonlítása}
\subsection{Virtuális gépek migrációja}
\subsection{Konténerizáció}

\section{Logikai kötetkezelés}
\subsection{Snapshotok, mentések készítése}


% TODO: másik fejezetbe + struktúra meghatározása
\chapter{Saját munka bemutatása}

\section{Kialakítani kívánt környezet meghatározása}

\section{Fizikai gép ismertetése}

\section{Operációs rendszer}
\subsection{SUSE} % Leap, Slowroll, TW
\subsection{Debian, Ubuntu}

\section{Gépmenedzsment: Salt}
\subsection{Más hasonló megoldások}

\section{Monitoring}

\section{Továbbfejlesztési lehetőségek}