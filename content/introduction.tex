% !TeX encoding = UTF-8
% !TeX spellcheck = hu_HU
%----------------------------------------------------------------------------
\chapter{\bevezetes}
%----------------------------------------------------------------------------

%A bevezető tartalmazza a diplomaterv-kiírás elemzését, történelmi előzményeit, a feladat indokoltságát (a motiváció leírását), az eddigi megoldásokat, és ennek tükrében a hallgató megoldásának összefoglalását.
%
%A bevezető szokás szerint a diplomaterv felépítésével záródik, azaz annak rövid leírásával, hogy melyik fejezet mivel foglalkozik.

Napjainkban az informatika és az internet életünk szerves részévé vált. A számos szolgáltatás folyamatos rendelkezésre állásának biztosítása és a megnövekedett forgalom kiszolgálása jó néhány új technológia kifejlesztését követelte meg.

Dolgozatomban elsősorban egy általános képet szeretnék adni arról, hogy milyen üzemeltetési kihívásokkal kell szembenéznünk, ha egy ilyen szolgáltatás működtetésébe vágjuk a fejszénket. Értekezésemben nem fogok kitérni bizonyos infrastrukturális hátterekre -- mint például a kiszolgálók folyamatos energiaellátásának biztosítása --, ezeket adottnak fogom tekinteni, hiszen ezt egy adatközpontban bérelt hely esetén sem magunknak kell biztosítanunk. A következőkben sokkal inkább az informatikai lehetőségek tárgyalására fogom helyezni a hangsúlyt: hogyan tudunk hatékonyan üzemeltetni több kiszolgálót, milyen módon lehet biztosítani a szolgáltatásaink lehető legnagyobb rendelkezésre állását, és hogyan védhetjük meg adatainkat egy esetlegesen félresikerült rendszerfrissítést követően.

A dolgozatban érinteni fogom a jelenleg legelterjedtebb virtualizációs technológiákat, melyek főbb tulajdonságait röviden ismertetem, valamint össze is hasonlítom ezeket a megoldásokat a legfontosabb különbségekre kitérve.

Szerepet fog kapni továbbá a logikai kötetkezelés, ezen belül is a Linux kernelben elérhető LVM-implementáció. Ez a technológia nagyban megkönnyíti a háttértárak és partíciók kezelését üzemeltetési szempontból, melyet főként virtualizációt végző fizikai gépek esetében használhatunk ki, hiszen ilyen helyzetekben érdemes minden virtuális rendszernek külön partíciót létrehozni, amelyek kezelése (pl. egy esetleges bővítés során) a hagyományos particionálási megoldásokkal sokkal összetettebb feladat lenne.

Szót ejtek a monitoring megoldásokról is, melyek elengedhetetlenek ahhoz, hogy a rendszer üzemeltetését végző szakemberek pontos képet kapjanak az infrastruktúra aktuális állapotáról, az esetleges korábbi problémákról. A monitorozás azért is fontos, mert ha egy hibát ezáltal sikerül idejekorán felismerni (például háttértárak esetén egy megfelelő határék beállításával időben értesülhetünk egy partíció megteléséről, és nem csak az írási hibákat tapasztaljuk), akkor elkerülhetőek a további, komolyabb hibák, amik akár a felhasználók számára is fennakadásokat okozhatnak. Az általam létrehozott tesztkörnyezetben is bemutatok egy ilyen monitoring megoldást, melynek segítségével az általam létrehozott infrastruktúra gépeit fogom folyamatosan ellenőrizni.

A tesztkörnyezet beállításában nagy szerep fog jutni a választott konfigurációmenedzsment szoftvernek, a Salt-nak. Ez arra fog lehetőséget biztosítani, hogy egyes konfigurációs fájlokat egyszerűen telepíthessünk több számítógépre is, valamint a keretrendszer leírónyelvén meghatározott konfigurációleíró szoftver lehetővé teszi azt is, hogy ellenőrizzük egyes szolgáltatások (service) állapotát. Ez hasznunkra válhat például egy saját service-szel érkező program telepítését követően, hiszen így a leíróban megadhatjuk a telepítés paramétereit, majd ezt követően egyből ellenőrizhetjük is, hogy a telepítés után sikeresen elindult-e az újonnan telepített szoftver.


A dolgozatban tárgyalt koncepciókat egy kisebb volumenű tesztrendszeren keresztül fogom bemutatni. Ennek a rendszernek a célja nem egy teljes vállalati környezet bemutatása, hiszen ehhez nagy mennyiségű hardverre, jelentős mértékű hardveres és szoftveres erőforrásokra lenne szükség, amelyek üzembe helyezése, összehangolása túlmutat a dolgozat keretein.
Ehelyett sokkal inkább arra szeretnék rávilágítani, hogy milyen eszközök állnak rendelkezésre egy ilyen nagyszabású infrastruktúra sikeres üzemeltetésének elősegítéséhez. Gondoljunk csak arra, hogy egy 5-10 számítógépből álló rendszer esetén kivitelezhető, hogy a rendszergazdák egyesével telepítsék a havi frissítéseket, azonban egy több száz, vagy több ezer kiszolgálóból álló nagyvállalati környezetben nem lenne egy reális elvárás.

Az ilyen és ehhez hasonló kihívások megoldására fogok lehetőségeket mutatni \aref{chap:technologies}.~fejezetben. Szó lesz a gépek távoli kezeléséről, folyamatos karbantartásukról, automatikus biztonsági javításokról (patchek) való értesülésről, ezek telepítéséről. Tárgyalni fogom továbbá a rendszert alkotó eszközök monitorozását, metrikák gyűjtését is, továbbá szó lesz az egyre szélesebb körben elterjedő konténerizációs technológiákról, ezek használatáról vállalati környezetekben. Bemutatom azt is, hogy a megfelelő eszközökkel milyen gyorsan hozhatunk létre konténereket, és mennyire hatékonyan kezelhetjük őket akár egy böngészőből is.
Fontos megjegyezni, hogy az itt említett technológiák kisebb környezetekben is használhatóak, azonban néhány esetben az ilyen rendszerek használata kevesebb előnyt nyújt, mint amennyi munkát telepítésük és karbantartásuk igényel, így érdemes felmérni az informatikai rendszerrel szemben támasztott elvárásainkat, és ennek megfelelően dönteni a szükséges technológiai komponensekről.

\Aref{chap:testenv}.~fejezetben fogom ismertetni az általam készített tesztkörnyezetet, ennek felépítését, a tervezési döntéseket, komponenseit, valamint az ezzel kapcsolatos munkáim során felmerült nehézségeket, tapasztalatokat. Ebben a fejezetben a korábban tárgyalt technológiák közül általam választott megoldásokat fogom részletesebben ismertetni.

Végül a dolgozat utolsó fejezetében értékelni fogom az elért eredményeket, valamint röviden összefoglalom a projekt továbbfejlesztési lehetőségeit.


