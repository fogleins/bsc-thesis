% !TeX encoding = UTF-8
% !TeX spellcheck = hu_HU

\chapter{Összefoglalás}

\section{Elért eredmények}
Dolgozatomban egy átfogó képet adtam a nagyvállalati rendszerek üzemeltetése során felmerülő kihívásokról, ezek lehetséges hatékony megoldásairól. Részletesen kitértem a kötetkezelés, az infrastruktúra-menedzsment és a monitoring témakörökre, melyek egy nagy méretű infrastruktúra üzemeltetésének elengedhetetlen kellékei. Ismertettem néhány konkrét megoldást, egy kis méretű tesztkörnyezetben bemutattam ezek alkalmazási lehetőségeit, a~tesztrendszer komponenseihez kapcsolódóan méréseket végeztem.

Munkám során megismerkedtem számos elterjedt technológiával, és mélyrehatóan tanulmányoztam ezeket. A téma jó dokumentáltsága lehetőséget nyújtott arra, hogy több különböző forrásból szerezzek információkat a megoldásokat illetően, és részletesen megismerkedjek egy-egy technológiával. Az ismeretek elsajátításához nagyban hozzájárultak a szoftvergyártók és a termékek hivatalos írásai, melyek magas minőségű tananyagoknak bizonyultak. A megismert lehetőségek megértésében nagy szerepet játszott a tesztkörnyezet, melyben alkalmam nyílt a megoldások kipróbálására, és szabadon kísérletezhettem a~különböző eszközökkel.

\section{Továbbfejlesztési lehetőségek}
Az értekezésem során részletesen tárgyaltam a kötetkezelés és az adattárolás témakörét, de~a~tesztkörnyezetben nem volt lehetőségem sem RAID-megoldás alkalmazására, sem pedig valamilyen biztonságimentés-szolgáltatás telepítésére vagy igénybe vételére. Úgy~gondolom, hogy ezek egy valós környezetben elengedhetetlenek, hiszen egy adatvesztés további súlyos problémákat okozhat, ezért fontos, hogy egy szervezet mindig fel legyen készülve az adatok visszaállítására, akár hardveres meghibásodás, akár rosszindulatú támadás okozta azt. Emiatt fontosnak tartom egy backup-keretrendszer megismerését, egy mentési stratégia kialakítását.

Ahogy már többször említettem a dolgozatom során, egy nagyvállalati környezetben nem ritka, hogy több száz vagy akár több ezer gépet kell kezelnünk. A tesztkörnyezetben ehhez képest mindössze három virtuális gépet és egy fizikai gépet üzemeltettem. Értelemszerűen nem reális, hogy egy ilyen jellegű környezetbe több száz gépet telepítsünk, de~úgy~gondolom, hogy például az infrastruktúra-menedzsment eszközök képességei jobban bemutathatóak lettek volna, ha néhány géppel többet tudok kezelni (pl.~egy külön webszervert, egy adatbázisszervert és egy levelezőkiszolgálót). Mindemellett ezen rendszerek konfigurálása és összehangolása jelentős többletmunkával járt volna, helyes beállításuk (ide~értve például a megfelelő titkosítási eljárások alkalmazását) körülbelül az eddig létrehozott környezettel azonos időt igényelt volna.

A lehetőségek tanulmányozása során nagyon tetszett Salt state-ek sokszínűsége, hogy egy pár soros leírófájllal mennyi mindent meg lehet oldani, viszonylag komplex dolgokat is. Biztos vagyok benne, hogy az itt érintettnél sokkal nagyobb mélységekbe lehet bocsátkozni a technológiát illetően, így a Salt-leírók továbbfejlesztése, szélesebb körű használata is jó~kihívásnak ígérkezik.

A monitoring rendszer rendkívül sok irányban továbbfejleszthető, például Grafanaban számtalan lehetőség áll rendelkezésünkre további adatok megjelenítésére, a Prometheusba bevonhatunk újabb exportereket, és az Alertmanager segítségével finomíthatjuk a riasztások kezelését (pl.~más platformokon is jelezzen, vagy beállíthatunk részletesebb szűrést).

Láthatjuk, hogy a technológia fejlődésével és a trendek folyamatos változásával --~például a manapság egyre népszerűbb konténerizációs megoldások népszerűségének növekedésével~-- a lehetőségek határtalanok, az eddigi munkám számos módon kiegészíthető.